% ______________________________________________________________________________
%
% DVG001 -- Introduktion till Linux och små nätverk
%                              Inlämningsuppgift #3
% ~~~~~~~~~~~~~~~~~~~~~~~~~~~~~~~~~~~~~~~~~~~~~~~~~
% Author:   Jonas Sjöberg
%           tel12jsg@student.hig.se
%
% Date:     2016-03-15 -- 2016-03-TODO
%
% License:  Creative Commons Attribution 4.0 International (CC BY 4.0)
%           <http://creativecommons.org/licenses/by/4.0/legalcode>
%           See LICENSE.md for additional licensing information.
% ______________________________________________________________________________

\documentclass[11pt,a4paper]{article}

\usepackage[utf8]{inputenc}
\inputencoding{utf8}
\usepackage[swedish]{babel}
\usepackage[swedish]{isodate}
\usepackage[T1]{fontenc}

\usepackage{lmodern}
\usepackage{fullpage}

\usepackage{csquotes}               % Behövs av biblatex
%\usepackage[natbib=true,
%            style=numeric,
%            backend=biber]{biblatex}
\usepackage[natbib=true,
            style=ieee,
            backend=biber]{biblatex}
\addbibresource{tex/refs.bib}

\usepackage[binary-units=true]{siunitx}
\usepackage{float}
\usepackage{textcomp}
\usepackage{url}
\usepackage{graphicx}
%\usepackage{amssymb}
%\usepackage{amsmath}
\usepackage{amsfonts}
\usepackage{graphicx}
%\usepackage{microtype}

\usepackage[pdfusetitle,
            bookmarks=true,
            bookmarksnumbered=true,
            bookmarksopen=false,
            breaklinks=false,
            pdfborder={0 0 0},
            backref=false,
            colorlinks=false]{hyperref}

\newcommand{\screenshot}[4]{
\begin{figure}[H]
\centering
%\includegraphics[width=\linewidth]{#1}
\includegraphics[height=8.0cm]{#1}
\caption[#2]{#3}
\label{#4}
\end{figure}
}

\usepackage{minted}
\usemintedstyle{bw}

\usepackage{verbatim}
\usepackage{fancyvrb}
\usepackage{listings}

\newmintedfile[shellcode]{bash}{
%bgcolor=mintedbackground,
%fontfamily=tt,
fontsize=\footnotesize,
linenos=true,
numberblanklines=true,
numbersep=12pt,
numbersep=5pt,
%gobble=0,
frame=lines,
%framerule=0.4pt,
framesep=2mm,
funcnamehighlighting=true,
tabsize=4,
obeytabs=false,
mathescape=false
samepage=false,
showspaces=false,
showtabs =false,
texcl=false,
}

\renewcommand\listingscaption{Programlistning}
\renewcommand\listoflistingscaption{Programlistningar}

\usepackage{booktabs}
\usepackage{longtable}

\usepackage{pdfpages}


\title{\textsc{DVG001}                         \\
       Introduktion till Linux och små nätverk \\
       Laboration 3}

\author{                                 \\
  Jonas Sjöberg                          \\
  860224-xxxx                            \\
  Högskolan i Gävle                      \\
  \texttt{tel12jsg@student.hig.se}       \\
  \texttt{https://github.com/jonasjberg} \\
}

\date{}

\begin{document}
  \maketitle

  \begin{center}
  \begin{tabular}{l r}
    Utförd: & \isodate \printdate{2016-03-15} -- \today \\
    Kursansvarig lärare: & Anders Jackson
  \end{tabular}
  \end{center}

  \begin{abstract}
    Laboration i kursen \emph{DVG001 -- Introduktion till Linux och små
    nätverk} som läses på distans via Högskolan i Gävle under vårterminen 2016.
    Laborationen behandlar hantering av användare och grupper, rättigheter och
    filåtkomst, installation och konfiguration av program, behandling av
    kataloger och filer, samt visualisering av filsystemet med skapandet av en
    trädstruktur.
  \end{abstract}

  \newpage
  %\hypersetup{linkcolor=black}
  \setcounter{tocdepth}{3}
  \tableofcontents

  \bigskip

  \listoffigures
  \listoftables
  \listoflistings

  \newpage
  % ______________________________________________________________________________
%
% DVG001 -- Introduktion till Linux och små nätverk
%                              Inlämningsuppgift #3
% ~~~~~~~~~~~~~~~~~~~~~~~~~~~~~~~~~~~~~~~~~~~~~~~~~
% Author:   Jonas Sjöberg
%           tel12jsg@student.hig.se
%
% Date:     2016-03-15 -- 2016-03-20
%
% License:  Creative Commons Attribution 4.0 International (CC BY 4.0)
%           <http://creativecommons.org/licenses/by/4.0/legalcode>
%           See LICENSE.md for additional licensing information.
% ______________________________________________________________________________


\section{Inledning}\label{inledning}
% Skriv en kort inledning här som beskriver kortfattat vad rapporten handlar
% om. Den skall vara orienterande om Bakgrund och Syfte.


% ______________________________________________________________________________
\subsection{Bakgrund}
% Beskriv lite mer ingående om bakgrunden till uppgiften, vad den handlar om.


% ______________________________________________________________________________
\subsection{Syfte}
% Skriv lite mer ingående om syftet med uppgiften.

% ______________________________________________________________________________
\subsection{Arbetsmetod}
% Hur kommer ni att arbeta?  Detta är en lite längre text än den rent
% orienterande texten i Planering och genomförande ovan.

Nedan följer en preliminär redogörelse för den experimentuppställning som används
under laborationen:

\begin{itemize}
  \item Laborationen utförs på en \texttt{ProBook-6545b} laptop som kör
        \texttt{Xubuntu 15.10} på kerneln \texttt{Linux 3.19.0-28-generic}.

  \item Rapporten skrivs i \LaTeX\  som kompileras till pdf med \texttt{latexmk}.

  \item Både rapporten och koden skrivs med texteditorn \texttt{Vim}.

  \item För versionshantering av både rapporten och programkod används \texttt{Git}.
    \begin{itemize}
      \item Källkod till programmet och rapporten finns att hämta på:

            \url{https://github.com/jonasjberg/DVG001\_lab3}

      \item Hämta hem repon genom att exekvera följande från kommandoraden:
            
            \texttt{git clone git@github.com:jonasjberg/DVG001\_lab3.git}

    \end{itemize}

  \item Virtualisering sker med \texttt{Oracle VirtualBox} version
        \texttt{5.0.10\_Ubuntu r104061}.
\end{itemize}



  %% ______________________________________________________________________________
%
% DVG001 -- Introduktion till Linux och små nätverk
%                              Inlämningsuppgift #3
% ~~~~~~~~~~~~~~~~~~~~~~~~~~~~~~~~~~~~~~~~~~~~~~~~~
% Author:   Jonas Sjöberg
%           tel12jsg@student.hig.se
%
% Date:     2016-03-15 -- 2016-03-TODO
%
% License:  Creative Commons Attribution 4.0 International (CC BY 4.0)
%           <http://creativecommons.org/licenses/by/4.0/legalcode>
%           See LICENSE.md for additional licensing information.
% ______________________________________________________________________________


\section{Genomförande}
% Här skriver ni vilka steg ni gjorde och resultatet av dem. Ni skall ha med
% information så att vi kan se hur ni har gjort, dvs beskrivande text,
% skärmdumpar, bilder etc.  Längre skärmdumpar, innehåll i relevanta filer och
% större bilder lägger ni i bilagor, som bilaga I, så att de inte tar över en
% sida själva.
% Kommandon som ni skriver i ett skal skall skrivas i detta format, som är
% teckenformatmall ”Exempel” i OpenOffice/LibreOffice. Detta så att de
% skiljer sig från övriga brödtext i stycket.  Detta underlättar läsningen för
% andra, som oss lärare.




% ______________________________________________________________________________
\subsection{Inledande åtgärder}

\subsubsection{Montera den delade mappen automatiskt}
Den delade mappen som konfigurerats tidigare under installationen av
gästsystemet har hittils monterats manuellt innan varje användning med
kommandot i Programlistning~\ref{listing:mount-share-manual}.

\begin{listing}[H]
  \shellcode{tex/mount-share-manual.sh}
  \caption{Kommando för att montera den delade mappen}
  \label{listing:mount-share-manual}
\end{listing}

Istället för att behöva göra det manuellt varje gång kan mappen läggas till i
\texttt{/etc/fstab} och monteras automatiskt vid systemstart.

Först görs en säkerhetskopia av \texttt{/etc/fstab} enligt
Programlistning~\ref{listing:mount-fstab-backup}.

\begin{listing}[H]
  \shellcode{tex/mount-fstab-backup.sh}
  \caption{Kommando för att skapa en säkerhetskopia av den befintliga filen.}
  \label{listing:mount-fstab-backup}
\end{listing}

Filen öppnas sedan i \texttt{Vim} med \texttt{root}-rättigheter genom
\texttt{sudo} och en extra rad läggs till enligt \cite{virtualbox:mntshare}.

Den ändrade filens innehåll visas
Programlistning~\ref{listing:mount-fstab-new}.

\begin{listing}[H]
  \shellcode{tex/mount-fstab-new.sh}
  \caption{Innehåll i filen \texttt{/etc/fstab} efter modifiering.}
  \label{listing:mount-fstab-new}
\end{listing}

För att få tillgång till den delade mapp som moteras automatiskt genom
\texttt{Guest Additions} måste användaren vara medlem av gruppen \texttt{vboxsf}


% Byt nätverkstyp i VirtualBox till bridged
% Kolla ip-adress, ifconfig deprecated (?)
% $ ip a
% ...



  % ______________________________________________________________________________
%
% DVG001 -- Introduktion till Linux och små nätverk
%                              Inlämningsuppgift #3
% ~~~~~~~~~~~~~~~~~~~~~~~~~~~~~~~~~~~~~~~~~~~~~~~~~
% Author:   Jonas Sjöberg
%           tel12jsg@student.hig.se
%
% Date:     2016-03-15 -- 2016-03-TODO
%
% License:  Creative Commons Attribution 4.0 International (CC BY 4.0)
%           <http://creativecommons.org/licenses/by/4.0/legalcode>
%           See LICENSE.md for additional licensing information.
% ______________________________________________________________________________


\section{Del ett}
Jag gick händelserna i förväg och gjorde under en del moment som inte ingick i
uppgiftsbeskrivningen för den förra laborationen, inlämningsuppgift #2.

Installation och konfiguration av \texttt{sudo} var ett mer eller mindre
nödvändigt steg i förberedelsen av den utvecklingsmiljö som skulle komma att
användas under laborationen.

Den beskrivning som följer är därför delvis hämtad direkt från föregående rapport.


% ______________________________________________________________________________
\subsection{Installation av \texttt{sudo}}
Programmet \texttt{sudo} installeras automatiskt om lösenord för
\texttt{root}-användaren inte specifieras under installationen av
\texttt{Debian}.  Av den anledningen behövde jag själv inte installera
\texttt{sudo}, bara konfiguration var nödvändig.

Från ''Debian GNU\slash Linux Installation Guide''
\cite{debian:install}:

\begin{quotation}
By default you are asked to provide a password for the “root” (administrator)
account and information necessary to create one regular user account. If you do
not specify a password for the “root” user this account will be disabled but
the sudo package will be installed later to enable administrative tasks to be
carried out on the new system. 
\end{quotation}


Programmet \texttt{sudo} ligger i paketet med samma namn.
För att ta redo på vilket paket som innehåller en viss fil kan man använda
kommandot i Programlistning~\ref{listing:dpkg-search}.

\begin{listing}[H]
\shellcode{tex/dpkg-search.sh}
\caption{Kommando för att söka efter filer i paket.}
\label{listing:dpkg-search}
\end{listing}


\subsection{Konfiguration av \texttt{sudo}}
För att kunna installera program och sköta andra administrativa sysslor
konfigureras \texttt{sudo} enligt följande:

\begin{enumerate}
  \item Användaren läggs till i \texttt{sudo}-gruppen med kommandot:

        \texttt{\$ adduser jonas sudo}

  \item Konfigurationsfilen för \texttt{sudo} ändras. Genom att köra
        följande kommandon så öppnar en särskild texteditor som tillåter en
        säkrare mijö för att ändra i filen \texttt{/etc/sudoers}:
        
        \texttt{\$ su} \\
        \texttt{\$ visudo}

        Med texteditorn lägger man till raden:
        \texttt{jonas ALL=(ALL:ALL) ALL}

        Det ger användaren \texttt{jonas} möjlighet att köra alla kommandon som 
        administratör med hjälp av \texttt{sudo}.
\end{enumerate}






% ______________________________________________________________________________
\subsection{Installation av \texttt{ntp}}
Här installeras programvara för att synkronisera olika datorers
tidsinställningar.  Det görs med \texttt{Debian}-paketet \texttt{ntp} som
ställer datorns klocka efter atomur som görs tillgängliga över internet genom
särskilda servrar.

\subsubsection{Installation av programmet \texttt{ntpdate} och \texttt{ntp}}
För att söka efter paketet \texttt{ntp} i lokala listor över paketförrådens
innehåll används kommandot i Programlistning~\ref{listing:ntp-apt-search},
som dessutom använder \texttt{grep} för att bara visa rader som börjar med
''ntp''.

\begin{listing}[H]
\shellcode{tex/ntp-apt-search.sh}
\caption{Kommando för att söka i lokala paketlistor efter textsträngar.}
\label{listing:ntp-apt-search}
\end{listing}


Programmen installeras sedan genom att köra kommandot i
Programlistning~\ref{listing:ntp-install}, som också visar information som skrivs
ut av \texttt{apt} under installationsprocessen.

\begin{listing}[H]
\shellcode{tex/ntp-install.sh}
\caption{Kommando för att installera programmen \texttt{ntp} och \texttt{ntpdate}.}
\label{listing:ntp-install}
\end{listing}


Användning av \texttt{apt}, \texttt{aptitude} eller
\texttt{dpkg} är många gånger utbytbar, både \texttt{apt} och \texttt{aptitude}
är abstraktioner byggda ovanpå \texttt{dpkg}. Även om \texttt{aptitude} ska vara
det nyaste och mest användarvänliga, föredrar användare många gånger ändå
att använda \texttt{apt}, av många olika skäl \cite{superuser:aptitude-apt}.

Efter installationen kan ett sökkommando motsvarande det i
Programlistning~\ref{listing:ntp-apt-search} köras, fast den här gången med
\texttt{aptitude}, som erbjuder inbyggd filtrering av resultat. Här visas
träffar som börjar med ''ntp''. Resultatet visas i
Programlistning~\ref{listing:ntp-aptitude-search}.

\begin{listing}[H]
\shellcode{tex/ntp-aptitude-search.sh}
\caption{Kommando för att söka bland installerade paket med \texttt{aptitude}.}
\label{listing:ntp-aptitude-search}
\end{listing}

\subsubsection{Konfiguration av \texttt{ntpdate} och \texttt{ntp}}
Programmen \texttt{ntpdate} och \texttt{ntp} konfigureras så att de använder
samma atomur-server på internet från ''ntp-pool''-projektet, exempelvis
\url{1.debian.pool.ntp.org} och \url{2.debian.pool.ntp.org} eller
\url{1.se.pool.ntp.org} och \url{2.se.pool.ntp.org}.
% Se <http://www.pool.ntp.org/>.

Efter installationen av programmen kan information som vanligt läsas från
manualsidorna. Konfigurationsfilen \texttt{/etc/ntp.conf} innehåller
standardinställningar, däribland en lista med servrar som används vid
tidssynkronisering. Ett utdrag visas i Programlistning~\ref{listing:ntp-conf}.
Raderna kan ändras för att innehålla de servrar vi själva vill använda,
förslagsvis svenska servrar vilket kan antas ge bättre tillförlitlighet och
snabbare respons. Ändringar görs med texteditorn \texttt{vim} som körs med
högre rättigheter med hjälp av \texttt{sudo}.

Nämnvärt är ordet \texttt{iburst} som är en \texttt{command option} som 
beskrivs i manualsidan för \texttt{ntp.conf(5)}.


\begin{listing}[H]
\configfile[firstline=18,lastline=24]{tex/ntp-config-default}
\caption{Utdrag ur den omodifierade konfigurationsfilen för \texttt{ntp}.}
\label{listing:ntp-conf}
\end{listing}

På ''ntp pool''-projektets sida \url{http://www.pool.ntp.org/zone/se} finns en
lista över svenska servrar tillsammans med statistik över deras aktivitet.
Enligt instruktioner läggs rader till i konfigurationsfilen, som då får
utseendet som visas i Programlistning~\ref{listing:ntp-conf-mod}.

\begin{listing}[H]
\configfile[firstline=18,lastline=28]{tex/ntp-config-modified}
\caption{Utdrag ur konfigurationsfilen för \texttt{ntp} efter inkludering av
servrar från en svensk ''pool zone''.}
\label{listing:ntp-conf-mod}
\end{listing}

Efter att konfigurationsfilen har modifierats startas \texttt{ntpd} om,
varpå en lista med ''peers'' som servern känner till skrivs ut. Detta
visas i Programlistning~\ref{listing:ntp-reload-list}.

\begin{listing}[H]
\shellcode{tex/ntp-reload-list}
\caption{Omstart av \texttt{ntpd} och listning av ''peers''.}
\label{listing:ntp-reload-list}
\end{listing}

  % ______________________________________________________________________________
%
% DVG001 -- Introduktion till Linux och små nätverk
%                              Inlämningsuppgift #3
% ~~~~~~~~~~~~~~~~~~~~~~~~~~~~~~~~~~~~~~~~~~~~~~~~~
% Author:   Jonas Sjöberg
%           tel12jsg@student.hig.se
%
% Date:     2016-03-15 -- 2016-03-20
%
% License:  Creative Commons Attribution 4.0 International (CC BY 4.0)
%           <http://creativecommons.org/licenses/by/4.0/legalcode>
%           See LICENSE.md for additional licensing information.
% ______________________________________________________________________________


\section{Del två}


% ______________________________________________________________________________
\subsection{Skapa en användare}
Här skapas en ny användare med kommandot \texttt{adduser}.  Skapandet av
användaren ''gibson'' visas i Programlistning~\ref{listing:adduser-gibson}.

\begin{listing}[H]
\shellcode{tex/adduser-gibson}
\caption{Skapande av en ny användare ''gibson''.}
\label{listing:adduser-gibson}
\end{listing}

Min katt Gibson har inte ett eget rum och använder för tillfället inte heller
någon telefon. En del av fälten lämnas därför tomma.


% ______________________________________________________________________________
\subsection{Logga in som den nya användaren}
Efter att den nya användaren har skapats så loggar man in som den nya
användaren.  

\subsubsection{Testkörning av \texttt{sudo} med den nya användaren}
En testkörning av \texttt{ifconfig} med \texttt{sudo} visar också
att användaren ''gibson'' inte har möjlighet att köra program med högre
privilegier ännu. Körningen visas i Programlistning~\ref{listing:su-gibson}.

\begin{listing}[H]
\shellcode{tex/su-gibson}
\caption[Logga in med annan användare]{Inloggning och testkörning av
         \texttt{sudo} med användaren ''gibson''.}
\label{listing:su-gibson}
\end{listing}


% ______________________________________________________________________________
\subsubsection{Lägga till den nya användaren i gruppen \texttt{sudo}}
För att ge den nya användaren möjlighet att köra \texttt{sudo} läggs den till i
gruppen som kallas \texttt{sudoers}. Detta visas i
Programlistning~\ref{listing:sudoers-gibson} tillsammans med en testkörning för
att verifiera att användaren fått möjlighet att köra kommandon som
\texttt{root} genom \texttt{sudo}.

\begin{listing}[H]
\shellcode{tex/sudoers-gibson}
\caption[Lägga till användare i gruppen \texttt{sudoers}]
        {Här inkluderas användaren ''gibson'' i gruppen \texttt{sudoers},
         följt av en testkörning.}
\label{listing:sudoers-gibson}
\end{listing}


Användaren ''gibson'' kan nu köra programmet \texttt{ifconfig} utan problem.

  % ______________________________________________________________________________
%
% DVG001 -- Introduktion till Linux och små nätverk
%                              Inlämningsuppgift #3
% ~~~~~~~~~~~~~~~~~~~~~~~~~~~~~~~~~~~~~~~~~~~~~~~~~
% Author:   Jonas Sjöberg
%           tel12jsg@student.hig.se
%
% Date:     2016-03-15 -- 2016-03-TODO
%
% License:  Creative Commons Attribution 4.0 International (CC BY 4.0)
%           <http://creativecommons.org/licenses/by/4.0/legalcode>
%           See LICENSE.md for additional licensing information.
% ______________________________________________________________________________


\section{Del tre}


% ______________________________________________________________________________
\subsection{Skapa katalogstruktur}
% TODO: Skapa användare, se till att användaren kan använda sudo.


% ______________________________________________________________________________
\subsection{Exekvering av kommandon}
% TODO: ..
% Skriv de kommandon som använts för att skapa filer, kataloger, rättigheter. 
%
% Förklara vad varje kommandorad gör och varför det händer, 
% samt vilken användare som utför delkommandon.
%
% Förklara felmeddelanden. 
%
% Katalogstrukturen redovisas lämpligen med kommandot `tree(1)` med växeln `-a`.


\subsubsection{Installera programmet \texttt{ntpdate} och \texttt{ntp}}
% TODO: ..

\subsubsection{Konfiguration av \texttt{ntpdate} och \texttt{ntp}}
% TODO: ..
%
% Konfigurera dem så att de använder samma atomur-server på internet från
% ntp-pool-projektet, exempelvis <1.debian.pool.ntp.org> och
% <2.debian.pool.ntp.org> eller <1.se.pool.ntp.org> och <2.se.pool.ntp.org>.
% Se <http://www.pool.ntp.org/>.





  \newpage
  % ______________________________________________________________________________
%
% DVG001 -- Introduktion till Linux och små nätverk
%                              Inlämningsuppgift #3
% ~~~~~~~~~~~~~~~~~~~~~~~~~~~~~~~~~~~~~~~~~~~~~~~~~
% Author:   Jonas Sjöberg
%           tel12jsg@student.hig.se
%
% Date:     2016-03-15 -- 2016-03-20
%
% License:  Creative Commons Attribution 4.0 International (CC BY 4.0)
%           <http://creativecommons.org/licenses/by/4.0/legalcode>
%           See LICENSE.md for additional licensing information.
% ______________________________________________________________________________


\section{Resultat}
Resultatet är en tydlig demonstration av hur rättigheter för användare och
grupper fungerar i \texttt{UNIX}-liknande system. En hel del vanligt förekommando
administrationssysslor har ingått i arbetet.

% ~~~~~~~~~~~~~~~~~~~~~~~~~~~~~~~~~~~~~~~~~~~~~~~~~~~~~~~~~~~~~~~~~~~~~~~~~~~~~~
\section{Diskussion}
Vissa lösningar kan vara något invecklade och möjligtvis onödigt komplexa,
men jag ser det som ett tillfälle att utveckla mina färdigheter dels i generell
systemadministration men också i allt som rör framställning av tekniska dokument
i \LaTeX , med allt som hör till att infoga källkod och kompilera dokument
automatiskt.


% ~~~~~~~~~~~~~~~~~~~~~~~~~~~~~~~~~~~~~~~~~~~~~~~~~~~~~~~~~~~~~~~~~~~~~~~~~~~~~~
\section{Slutsatser}
Jag stötte inte på några problem under utförandet, allt verkar ha gått som
planerat.


  \addcontentsline{toc}{section}{Referenser}
  \printbibliography{}

\end{document}
