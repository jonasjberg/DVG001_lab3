% ______________________________________________________________________________
%
% DVG001 -- Introduktion till Linux och små nätverk
%                              Inlämningsuppgift #3
% ~~~~~~~~~~~~~~~~~~~~~~~~~~~~~~~~~~~~~~~~~~~~~~~~~
% Author:   Jonas Sjöberg
%           tel12jsg@student.hig.se
%
% Date:     2016-03-15 -- 2016-03-TODO
%
% License:  Creative Commons Attribution 4.0 International (CC BY 4.0)
%           <http://creativecommons.org/licenses/by/4.0/legalcode>
%           See LICENSE.md for additional licensing information.
% ______________________________________________________________________________

\documentclass[11pt,a4paper]{article}

\usepackage[utf8]{inputenc}
\inputencoding{utf8}
\usepackage[swedish]{babel}
\usepackage[swedish]{isodate}
\usepackage[T1]{fontenc}

\usepackage{lmodern}
\usepackage{fullpage}

\usepackage{csquotes}               % Behövs av biblatex
%\usepackage[natbib=true,
%            style=numeric,
%            backend=biber]{biblatex}
\usepackage[natbib=true,
            style=ieee,
            backend=biber]{biblatex}
\addbibresource{tex/refs.bib}

\usepackage[binary-units=true]{siunitx}
\usepackage{float}
\usepackage{textcomp}
\usepackage{url}
\usepackage{graphicx}
%\usepackage{amssymb}
%\usepackage{amsmath}
\usepackage{amsfonts}
\usepackage{graphicx}
%\usepackage{microtype}

\usepackage[pdfusetitle,
            bookmarks=true,
            bookmarksnumbered=true,
            bookmarksopen=false,
            breaklinks=false,
            pdfborder={0 0 0},
            backref=false,
            colorlinks=false]{hyperref}

\newcommand{\screenshot}[4]{
\begin{figure}[H]
\centering
%\includegraphics[width=\linewidth]{#1}
\includegraphics[height=8.0cm]{#1}
\caption[#2]{#3}
\label{#4}
\end{figure}
}

\usepackage{minted}
\usemintedstyle{bw}

\usepackage{verbatim}
\usepackage{fancyvrb}
\usepackage{listings}

\newmintedfile[shellcode]{bash}{
%bgcolor=mintedbackground,
%fontfamily=tt,
fontsize=\footnotesize,
linenos=true,
numberblanklines=true,
numbersep=12pt,
numbersep=5pt,
%gobble=0,
frame=lines,
%framerule=0.4pt,
framesep=2mm,
funcnamehighlighting=true,
tabsize=4,
obeytabs=false,
mathescape=false
samepage=false,
showspaces=false,
showtabs =false,
texcl=false,
}

\renewcommand\listingscaption{Programlistning}
\renewcommand\listoflistingscaption{Programlistningar}

\usepackage{booktabs}
\usepackage{longtable}

\usepackage{pdfpages}


\title{\textsc{DVG001}                         \\
       Introduktion till Linux och små nätverk \\
       Laboration 3}

\author{                                 \\
  Jonas Sjöberg                          \\
  860224-xxxx                            \\
  Högskolan i Gävle                      \\
  \texttt{tel12jsg@student.hig.se}       \\
  \texttt{https://github.com/jonasjberg} \\
}

\date{}

\begin{document}
  \maketitle

  \begin{center}
  \begin{tabular}{l r}
    Utförd: & \isodate \printdate{2016-03-15} -- \today \\
    Kursansvarig lärare: & Anders Jackson
  \end{tabular}
  \end{center}

  \begin{abstract}
    Laboration i kursen \emph{DVG001 -- Introduktion till Linux och små
    nätverk} som läses på distans via Högskolan i Gävle under vårterminen 2016.
    Laborationen behandlar hantering av användare och grupper, rättigheter och
    filåtkomst, installation och konfiguration av program, behandling av
    kataloger och filer, samt visualisering av filsystemet med skapandet av en
    trädstruktur.
  \end{abstract}

  \newpage
  %\hypersetup{linkcolor=black}
  \setcounter{tocdepth}{3}
  \tableofcontents

  \bigskip

  \listoffigures
  \listoftables
  \listoflistings

  \newpage
  % ______________________________________________________________________________
%
% DVG001 -- Introduktion till Linux och små nätverk
%                              Inlämningsuppgift #3
% ~~~~~~~~~~~~~~~~~~~~~~~~~~~~~~~~~~~~~~~~~~~~~~~~~
% Author:   Jonas Sjöberg
%           tel12jsg@student.hig.se
%
% Date:     2016-03-15 -- 2016-03-TODO
%
% License:  Creative Commons Attribution 4.0 International (CC BY 4.0)
%           <http://creativecommons.org/licenses/by/4.0/legalcode>
%           See LICENSE.md for additional licensing information.
% ______________________________________________________________________________


\section{Inledning}\label{inledning}
% Skriv en kort inledning här som beskriver kortfattat vad rapporten handlar
% om. Den skall vara orienterande om Bakgrund och Syfte.


% ______________________________________________________________________________
\subsection{Bakgrund}
% Beskriv lite mer ingående om bakgrunden till uppgiften, vad den handlar om.


% ______________________________________________________________________________
\subsection{Syfte}
% Skriv lite mer ingående om syftet med uppgiften.

% ______________________________________________________________________________
\subsection{Arbetsmetod}
% Hur kommer ni att arbeta?  Detta är en lite längre text än den rent
% orienterande texten i Planering och genomförande ovan.

Nedan följer en preliminär redogörelse för den experimentuppställning som används
under laborationen:

\begin{itemize}
  \item Laborationen utförs på en \texttt{ProBook-6545b} laptop som kör
        \texttt{Xubuntu 15.10} på kerneln \texttt{Linux 3.19.0-28-generic}.

  \item Rapporten skrivs i \LaTeX\  som kompileras till pdf med \texttt{latexmk}.

  \item Både rapporten och koden skrivs med texteditorn \texttt{Vim}.

  \item För versionshantering av både rapporten och programkod används \texttt{Git}.
    \begin{itemize}
      \item Källkod till programmet och rapporten finns att hämta på:

            \url{https://github.com/jonasjberg/DVG001\_lab3}

      \item Hämta hem repon genom att exekvera följande från kommandoraden:
            
            \texttt{git clone git@github.com:jonasjberg/DVG001\_lab3.git}

    \end{itemize}

  \item Virtualisering sker med \texttt{Oracle VirtualBox} version
        \texttt{5.0.10\_Ubuntu r104061}.
\end{itemize}



  %% ______________________________________________________________________________
%
% DVG001 -- Introduktion till Linux och små nätverk
%                              Inlämningsuppgift #3
% ~~~~~~~~~~~~~~~~~~~~~~~~~~~~~~~~~~~~~~~~~~~~~~~~~
% Author:   Jonas Sjöberg
%           tel12jsg@student.hig.se
%
% Date:     2016-03-15 -- 2016-03-TODO
%
% License:  Creative Commons Attribution 4.0 International (CC BY 4.0)
%           <http://creativecommons.org/licenses/by/4.0/legalcode>
%           See LICENSE.md for additional licensing information.
% ______________________________________________________________________________


\section{Genomförande}
% Här skriver ni vilka steg ni gjorde och resultatet av dem. Ni skall ha med
% information så att vi kan se hur ni har gjort, dvs beskrivande text,
% skärmdumpar, bilder etc.  Längre skärmdumpar, innehåll i relevanta filer och
% större bilder lägger ni i bilagor, som bilaga I, så att de inte tar över en
% sida själva.
% Kommandon som ni skriver i ett skal skall skrivas i detta format, som är
% teckenformatmall ”Exempel” i OpenOffice/LibreOffice. Detta så att de
% skiljer sig från övriga brödtext i stycket.  Detta underlättar läsningen för
% andra, som oss lärare.




% ______________________________________________________________________________
\subsection{Inledande åtgärder}

\subsubsection{Montera den delade mappen automatiskt}
Den delade mappen som konfigurerats tidigare under installationen av
gästsystemet har hittils monterats manuellt innan varje användning med
kommandot i Programlistning~\ref{listing:mount-share-manual}.

\begin{listing}[H]
  \shellcode{tex/mount-share-manual.sh}
  \caption{Kommando för att montera den delade mappen}
  \label{listing:mount-share-manual}
\end{listing}

Istället för att behöva göra det manuellt varje gång kan mappen läggas till i
\texttt{/etc/fstab} och monteras automatiskt vid systemstart
\cite{virtualbox:sharemount}.

Först görs en säkerhetskopia av \texttt{/etc/fstab} enligt
Programlistning~\ref{listing:mount-fstab-backup}.



% Byt nätverkstyp i VirtualBox till bridged
% Kolla ip-adress, ifconfig deprecated (?)
% $ ip a 
% ...



  % ______________________________________________________________________________
%
% DVG001 -- Introduktion till Linux och små nätverk
%                              Inlämningsuppgift #3
% ~~~~~~~~~~~~~~~~~~~~~~~~~~~~~~~~~~~~~~~~~~~~~~~~~
% Author:   Jonas Sjöberg
%           tel12jsg@student.hig.se
%
% Date:     2016-03-15 -- 2016-03-TODO
%
% License:  Creative Commons Attribution 4.0 International (CC BY 4.0)
%           <http://creativecommons.org/licenses/by/4.0/legalcode>
%           See LICENSE.md for additional licensing information.
% ______________________________________________________________________________


\section{Del ett}
Jag gick händelserna i förväg och gjorde under en del moment som inte ingick i
uppgiftsbeskrivningen för den förra laborationen, inlämningsuppgift #2.

Installation och konfiguration av \texttt{sudo} var ett mer eller mindre
nödvändigt steg i förberedelsen av den utvecklingsmiljö som skulle komma att
användas under laborationen.

Den beskrivning som följer är delvis hämtad direkt från föregående rapport.


% ______________________________________________________________________________
\subsection{Installera \texttt{sudo}}
För att kunna installera program och sköta andra administrativa sysslor
konfigureras \texttt{sudo} enligt följande:

\begin{enumerate}
  \item Användaren läggs till i \texttt{sudo}-gruppen med kommandot:
        \texttt{\$ adduser jonas sudo}

  \item Konfigurationsfilen för \texttt{sudo} ändras. Genom att köra
        följande kommandon så öppnar en särskild texteditor som tillåter en
        säkrare mijö för att ändra i filen \texttt{/etc/sudoers}:
        
        \texttt{\$ su} \\
        \texttt{\$ visudo}

        Med texteditorn lägger man till raden:
        \texttt{jonas ALL=(ALL:ALL) ALL}

        Det ger användaren \texttt{jonas} möjlighet att köra alla kommandon som 
        administratör med hjälp av \texttt{sudo}.
\end{enumerate}

Programmet \texttt{sudo} ligger i paketet med samma namn.
För att ta redo på vilket paket som innehåller en viss fil kan man använda
kommandot i Programlistning~\ref{listing:dpkg-search}.

\begin{listing}[H]
\shellcode{tex/dpkg-search.sh}
\caption{Kommando för att söka efter filer i paket.}
\label{listing:dpkg-search}
\end{listing}

Programmet \texttt{sudo} installeras automatiskt om lösenord för
\texttt{root}-användaren inte specifieras under installationen av
\texttt{Debian}. Från ''Debian GNU\slash Linux Installation Guide''
\cite{debian:install}:

\begin{quotation}
By default you are asked to provide a password for the “root” (administrator)
account and information necessary to create one regular user account. If you do
not specify a password for the “root” user this account will be disabled but
the sudo package will be installed later to enable administrative tasks to be
carried out on the new system. 
\end{quotation}

% ______________________________________________________________________________
\subsection{Installera \texttt{ntp}}
Här installeras programvara för att synkronisera olika datorers
tidsinställningar.  Det görs med \texttt{Debian}-paketet \texttt{ntp} som
ställer datorns klocka efter atomur, särskilda servrar som agerar
tidsstandarder och finns tillgängliga över internet.

\subsubsection{Installera programmet \texttt{ntpdate} och \texttt{ntp}}
% TODO: ..

\subsubsection{Konfiguration av \texttt{ntpdate} och \texttt{ntp}}
% TODO: ..
%
% Konfigurera dem så att de använder samma atomur-server på internet från
% ntp-pool-projektet, exempelvis <1.debian.pool.ntp.org> och
% <2.debian.pool.ntp.org> eller <1.se.pool.ntp.org> och <2.se.pool.ntp.org>.
% Se <http://www.pool.ntp.org/>.


  % ______________________________________________________________________________
%
% DVG001 -- Introduktion till Linux och små nätverk
%                              Inlämningsuppgift #3
% ~~~~~~~~~~~~~~~~~~~~~~~~~~~~~~~~~~~~~~~~~~~~~~~~~
% Author:   Jonas Sjöberg
%           tel12jsg@student.hig.se
%
% Date:     2016-03-15 -- 2016-03-TODO
%
% License:  Creative Commons Attribution 4.0 International (CC BY 4.0)
%           <http://creativecommons.org/licenses/by/4.0/legalcode>
%           See LICENSE.md for additional licensing information.
% ______________________________________________________________________________


\section{Del två}


% ______________________________________________________________________________
\subsection{Skapa en användare}
Här skapas en ny användare med kommandot \texttt{adduser}.  Skapandet av
användaren ''gibson'' visas i Programlistning~\ref{listing:adduser-gibson}.

\begin{listing}[H]
\shellcode{tex/adduser-gibson}
\caption{Skapande av en ny användare ''gibson''.}
\label{listing:adduser-gibson}
\end{listing}

Min katt Gibson har inte ett eget rum och använder för tillfället inte heller
någon telefon. En del av fälten lämnas därför tomma.


% ______________________________________________________________________________
\subsection{Logga in som den nya användaren}
Efter att den nya användaren har skapats så loggar man in som den nya
användaren.  

\subsubsection{Testkörning av \texttt{sudo} med den nya användaren}
En testkörning av \texttt{ifconfig} med \texttt{sudo} visar också
att användaren ''gibson'' inte har möjlighet att köra program med högre
privilegier ännu. Körningen visas i Programlistning~\ref{listing:su-gibson}.

\begin{listing}[H]
\shellcode{tex/su-gibson}
\caption{Inloggning och testkörning av \texttt{sudo} med användaren ''gibson''.}
\label{listing:su-gibson}
\end{listing}


% ______________________________________________________________________________
\subsubsection{Lägga till den nya användaren i gruppen \texttt{sudo}}
För att ge den nya användaren möjlighet att köra \texttt{sudo} läggs den till i
gruppen som kallas \texttt{sudoers}. Detta visas i
Programlistning~\ref{listing:sudoers-gibson} tillsammans med en testkörning för
att verifiera att användaren fått möjlighet att köra kommandon som
\texttt{root} genom \texttt{sudo}.

\begin{listing}[H]
\shellcode{tex/sudoers-gibson}
\caption{Inkludering av användaren ''gibson'' i gruppen \texttt{sudoers} och testkörning.}
\label{listing:sudoers-gibson}
\end{listing}


Användaren ''gibson'' kan nu köra programmet \texttt{ifconfig} utan problem.

  % ______________________________________________________________________________
%
% DVG001 -- Introduktion till Linux och små nätverk
%                              Inlämningsuppgift #3
% ~~~~~~~~~~~~~~~~~~~~~~~~~~~~~~~~~~~~~~~~~~~~~~~~~
% Author:   Jonas Sjöberg
%           tel12jsg@student.hig.se
%
% Date:     2016-03-15 -- 2016-03-20
%
% License:  Creative Commons Attribution 4.0 International (CC BY 4.0)
%           <http://creativecommons.org/licenses/by/4.0/legalcode>
%           See LICENSE.md for additional licensing information.
% ______________________________________________________________________________


\section{Del tre}


% ______________________________________________________________________________
\subsection{Skapa katalogstruktur}
Här ska rättigheter för användare demonstreras genom skapande av en uppsättning filer
och kataloger. Den katalogstruktur som efterfrågas visas i Tabell~\ref{table:tree}.

\begin{table}[]
  \centering
  \caption{Efterfrågad katalogstruktur}
  \label{table:tree}
  \begin{tabular}{@{}llll@{}}
  \toprule
          Sökväg        & Rättigheter                   & Ägare                    & Grupp                          \\ \midrule
  \texttt{/tmp/del3}    & \texttt{drwxr-xr-x}           & Användaren från del två  & Gruppen som användaren tillhör \\
  \texttt{/tmp/del3/a1} & \texttt{drwx-{}-{}-{}-{}-{}-} & \texttt{root}            & Gruppen som användaren tillhör \\
  \texttt{/tmp/del3/a2} & \texttt{-rwxr-{}-r-{}-}       &  Användaren från del två & \texttt{root}                  \\
  \texttt{/tmp/del3/a3} & \texttt{drwxr-{}-r-{}-}       &  Användaren från del två & Gruppen som användaren tillhör \\
  \texttt{/tmp/del3/a4} & \texttt{-rwxrwx-{}-{}-}       & \texttt{root}            & \texttt{root}                  \\\bottomrule
  \end{tabular}
\end{table}



Katalogen \texttt{/tmp/del3/a1} skapas enligt Programlistning~\ref{listing:part3-del3a1}.

\begin{listing}[H]
\shellcode{tex/part3-del3a1}
\caption{Skapandet av \texttt{/tmp/del3/a1}}
\label{listing:part3-del3a1}
\end{listing}


Filen \texttt{/tmp/del3/a2} skapas enligt Programlistning~\ref{listing:part3-del3a2}.

\begin{listing}[H]
\shellcode{tex/part3-del3a2}
\caption{Skapandet av \texttt{/tmp/del3/a2}}
\label{listing:part3-del3a2}
\end{listing}


Katalogen \texttt{/tmp/del3/a3} skapas enligt Programlistning~\ref{listing:part3-del3a3}.

\begin{listing}[H]
\shellcode{tex/part3-del3a3}
\caption{Skapandet av \texttt{/tmp/del3/a3}}
\label{listing:part3-del3a3}
\end{listing}


Filen \texttt{/tmp/del3/a4} skapas enligt Programlistning~\ref{listing:part3-del3a4}.

\begin{listing}[H]
\shellcode{tex/part3-del3a4}
\caption{Skapandet av \texttt{/tmp/del3/a4}}
\label{listing:part3-del3a4}
\end{listing}


% ______________________________________________________________________________
\subsubsection{Verifiering av de filer som skapats i \texttt{/tmp/del3}}
För att kontrollera att de skapade filerna och katalogerna har de egenskaper
som efterfrågats körs programmet i Programlistning~\ref{listing:part3-verify}.
Programmet loopar över både innehållet i katalogen \texttt{/tmp/del3} och
katalogen själv med en möjligen onödigt komplex parameter-expansion
\texttt{\{,*\}} som expanderas till katalogen och dess underkataloger.  Dessa
skickas som argument till \texttt{ls} vars utskrift slutligen kolumniseras med
\texttt{column}.

\begin{listing}[H]
\shellcode{tex/part3-verify.sh}
\caption{Verifiering av de kataloger och filer som skapats}
\label{listing:part3-verify}
\end{listing}

Resultatet i Programlistning~\ref{listing:part3-verify} matchar den
struktur som efterfrågas i Tabell~\ref{table:tree}.


% ______________________________________________________________________________
\subsubsection{Skript för skapande av katalogstruktur}
Eftersom att katalogstrukturen ligger i \texttt{/tmp/} så är det stor risk att
den raderas när systemet startar om. Ovanstående kommandon samlas i ett skript
så att katalogstrukturen enkelt kan återskapas efter en omstart.

Skriptet visas i Programlistning~\ref{listing:part3-setup} och körning visas i
Programlistning~\ref{listing:part3-setup-output}.

\begin{listing}[H]
\shellcode{tex/part3-setup.sh}
\caption[Skript för att skapa katalogstruktur]{Skript som körs för att skapa
         filer och kataloger med särskilda rättigheter.}
\label{listing:part3-setup}
\end{listing}

\begin{listing}[H]
\shellcode{tex/part3-setup-output}
\caption{Körning av skriptet i Programlistning~\ref{listing:part3-setup}.}
\label{listing:part3-setup-output}
\end{listing}


% ______________________________________________________________________________
\subsection{Exekvering av kommandon}
% Förklara vad varje kommandorad gör och varför det händer, 
% samt vilken användare som utför delkommandon.
%
% Förklara felmeddelanden. 
%
% Katalogstrukturen redovisas lämpligen med kommandot `tree(1)` med växeln `-a`.

\newcommand{\explainedcmd}[4]{
\begin{listing}[H]
\shellcode[firstline={#1},lastline={#2}]{tex/part3-commands}
\caption[#3]{#4}
\label{listing:ntp-conf-mod}
\end{listing}
}

Under de Programlistningar som följer står förklaringar av vad varje kommando
gör och varför det händer skrivna under respektive Programlistning.

\explainedcmd{1}{3}
             {Kommandot \texttt{touch /tmp/del3/a1/f1}}
             {Kommandot misslyckas på grund av att enbart ägaren \texttt{root}
              har behörighet att skriva i målkatalogen.  
              Vid den andra körningen används \texttt{sudo} för att anta 
              \texttt{roots} rättigheter och kommandot lyckas.}

\explainedcmd{4}{6}
             {Kommandot \texttt{sudo echo "Hello World" | tee /tmp/del3/a3/f1}}
             {Körningen av \texttt{tee} misslyckas eftersom att ''gibson''
              saknar rättigheter. Bara \texttt{echo} körs med \texttt{sudo},
              rättigheterna ''nollställs'' då datan passerar pipe-symbolen,
              \texttt{|}.}

\explainedcmd{7}{8}
             {Kommandot \texttt{sudo echo "Hello World" | tee /tmp/del3/a2}}
             {Körningen av \texttt{tee} lyckas och ''Hello World'' skrivs
              till filen \texttt{a2}, som ägs av ''gibson''.}

\explainedcmd{9}{11}
             {Kommandot \texttt{sudo echo "Hello World" | tee /tmp/del3/a4}}
             {Körningen av \texttt{tee} misslyckas eftersom att ''gibson''
              saknar rättigheter för att skriva till filen \texttt{a4}. 
              Enbart \texttt{echo} körs med \texttt{sudo}, rättigheterna
              ''nollställs'' då datan passerar pipe-symbolen, \texttt{|}.}

\explainedcmd{12}{13}
             {Kommandot \texttt{cat /tmp/del3/a2}}
             {Innehållet i filen \texttt{a2} skrivs ut.}

\explainedcmd{14}{15}
             {Kommandot \texttt{sudo cat /tmp/del3/a2}}
             {Innehållet i filen \texttt{a2} skrivs ut. Användningen av
              \texttt{sudo} är överflödig, ''gibson'' har redan rättighet att
              läsa filen.}

\explainedcmd{16}{17}
             {Kommandot \texttt{cat /tmp/del3/a4}}
             {Körningen misslyckas eftersom att filen \texttt{a4} ägs av
              \texttt{root} och att ''gibson'' är inte medlem i gruppen 
              \texttt{root} och saknar därmed rättigheter.}

\explainedcmd{18}{18}
             {Kommandot \texttt{sudo cat /tmp/del3/a4}}
             {Körningen lyckas men filen \texttt{a4} är tom då kommandot för
              att skriva till filen misslyckades. Den text som skrevs ut för
              att beskriva felet skrevs till \texttt{stderror}, medan 
              \texttt{stdout} som skrevs till filen, var tomt.}

\explainedcmd{19}{20}
             {Kommandot \texttt{echo "Goodbye World" | sudo tee /tmp/del3/a2}}
             {Körningen lyckas trots att ''gibson'' inte har rättighet att
              skriva till \texttt{a2} då kommandot körs som \texttt{root} med
              \texttt{sudo}.}

\explainedcmd{21}{22}
             {Kommandot \texttt{cat /tmp/del3/a2}}
             {Körningen lyckas, ''gibson'' har rättighet att läsa \texttt{a2}.}

\explainedcmd{23}{24}
             {Kommandot \texttt{sudo cat /tmp/del3/a2}}
             {Körningen lyckas, ''gibson'' har rättighet att läsa \texttt{a2}.
              Användningen av \texttt{sudo} är överflödig.}

\explainedcmd{25}{26}
             {Kommandot \texttt{sudo rm -r /tmp/del3/}}
             {Hela katalogstrukturen tas bort.}

\subsection{Redovisning av resultat}
För att kontrollera och redovisa katalogstrukturens innehåll används ännu ett
skript som visas i Programlistning~\ref{listing:part3-treeish}.

Exekvering visas i Programlistning~\ref{listing:part3-treeish-output}. Kataloger
och filer skrivs ut och märks i den vänstra kolumnen, filers innehåll skrivs ut
i den högra kolumnen. Notera att det sista kommandot \texttt{sudo rm -r /tmp/del3/}
har exkluderats för att det ska vara möjligt att titta på resultatet.

\begin{listing}[H]
\shellcode{tex/treeish.sh}
\caption[Skript för att skapa katalogstruktur med rättigheter]{Skript som körs
         för att skapa filer och kataloger med särskilda rättigheter.}
\label{listing:part3-treeish}
\end{listing}

\begin{listing}[H]
\shellcode{tex/treeish-output}
\caption{Körning av skriptet i Programlistning~\ref{listing:part3-treeish}.}
\label{listing:part3-treeish-output}
\end{listing}


  \newpage
  % ______________________________________________________________________________
%
% DVG001 -- Introduktion till Linux och små nätverk
%                              Inlämningsuppgift #3
% ~~~~~~~~~~~~~~~~~~~~~~~~~~~~~~~~~~~~~~~~~~~~~~~~~
% Author:   Jonas Sjöberg
%           tel12jsg@student.hig.se
%
% Date:     2016-03-15 -- 2016-03-TODO
%
% License:  Creative Commons Attribution 4.0 International (CC BY 4.0)
%           <http://creativecommons.org/licenses/by/4.0/legalcode>
%           See LICENSE.md for additional licensing information.
% ______________________________________________________________________________


\section{Resultat}


% ~~~~~~~~~~~~~~~~~~~~~~~~~~~~~~~~~~~~~~~~~~~~~~~~~~~~~~~~~~~~~~~~~~~~~~~~~~~~~~
\section{Diskussion}
% Diskutera lite friare om vad resultatet betyder samt vad ni mer lärt er.
% Även om hur ni kanske skulle gjort annorlunda om ni gjort om det.


% ~~~~~~~~~~~~~~~~~~~~~~~~~~~~~~~~~~~~~~~~~~~~~~~~~~~~~~~~~~~~~~~~~~~~~~~~~~~~~~
\section{Slutsatser}
% Sammanfatta vad ni fått för resultat, utgå från syftet som ni angett i
% Syfte ovan.


  \addcontentsline{toc}{section}{Referenser}
  \printbibliography{}

\end{document}
