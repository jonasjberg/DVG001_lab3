% ______________________________________________________________________________
%
% DVG001 -- Introduktion till Linux och små nätverk
%                              Inlämningsuppgift #3
% ~~~~~~~~~~~~~~~~~~~~~~~~~~~~~~~~~~~~~~~~~~~~~~~~~
% Author:   Jonas Sjöberg
%           tel12jsg@student.hig.se
%
% Date:     2016-03-15 -- 2016-03-TODO
%
% License:  Creative Commons Attribution 4.0 International (CC BY 4.0)
%           <http://creativecommons.org/licenses/by/4.0/legalcode>
%           See LICENSE.md for additional licensing information.
% ______________________________________________________________________________


\section{Del ett}
Jag gick händelserna i förväg och gjorde under en del moment som inte ingick i
uppgiftsbeskrivningen för den förra laborationen, inlämningsuppgift #2.

Installation och konfiguration av \texttt{sudo} var ett mer eller mindre
nödvändigt steg i förberedelsen av den utvecklingsmiljö som skulle komma att
användas under laborationen.

Den beskrivning som följer är delvis hämtad direkt från föregående rapport.


% ______________________________________________________________________________
\subsection{Installera \texttt{sudo}}
För att kunna installera program och sköta andra administrativa sysslor
konfigureras \texttt{sudo} enligt följande:

\begin{enumerate}
  \item Användaren läggs till i \texttt{sudo}-gruppen med kommandot:
        \texttt{\$ adduser jonas sudo}

  \item Konfigurationsfilen för \texttt{sudo} ändras. Genom att köra
        följande kommandon så öppnar en särskild texteditor som tillåter en
        säkrare mijö för att ändra i filen \texttt{/etc/sudoers}:
        
        \texttt{\$ su} \\
        \texttt{\$ visudo}

        Med texteditorn lägger man till raden:
        \texttt{jonas ALL=(ALL:ALL) ALL}

        Det ger användaren \texttt{jonas} möjlighet att köra alla kommandon som 
        administratör med hjälp av \texttt{sudo}.
\end{enumerate}

Programmet \texttt{sudo} ligger i paketet med samma namn.
För att ta redo på vilket paket som innehåller en viss fil kan man använda
kommandot i Programlistning~\ref{listing:dpkg-search}.

\begin{listing}[H]
\shellcode{tex/dpkg-search.sh}
\caption{Kommando för att söka efter filer i paket.}
\label{listing:dpkg-search}
\end{listing}

Programmet \texttt{sudo} installeras automatiskt om lösenord för
\texttt{root}-användaren inte specifieras under installationen av
\texttt{Debian}.  Av den anledningen behövde jag själv inte installera
\texttt{sudo}, bara konfiguration var nödvändig.

Från ''Debian GNU\slash Linux Installation Guide''
\cite{debian:install}:

\begin{quotation}
By default you are asked to provide a password for the “root” (administrator)
account and information necessary to create one regular user account. If you do
not specify a password for the “root” user this account will be disabled but
the sudo package will be installed later to enable administrative tasks to be
carried out on the new system. 
\end{quotation}




% ______________________________________________________________________________
\subsection{Installera \texttt{ntp}}
Här installeras programvara för att synkronisera olika datorers
tidsinställningar.  Det görs med \texttt{Debian}-paketet \texttt{ntp} som
ställer datorns klocka efter atomur, särskilda servrar som agerar
tidsstandarder och finns tillgängliga över internet.

\subsubsection{Installera programmet \texttt{ntpdate} och \texttt{ntp}}
För att söka efter paketet \texttt{ntp} i lokala listor över paketförrådens
innehåll används kommandot i Programlistning~\ref{listing:ntp-apt-search},
som dessutom använder \texttt{grep} för att bara visa rader som börjar med
''ntp''.

\begin{listing}[H]
\shellcode{tex/ntp-apt-search.sh}
\caption{Kommando för att söka efter filer i paket.}
\label{listing:ntp-apt-search}
\end{listing}

\subsubsection{Konfiguration av \texttt{ntpdate} och \texttt{ntp}}
% TODO: ..
%
% Konfigurera dem så att de använder samma atomur-server på internet från
% ntp-pool-projektet, exempelvis <1.debian.pool.ntp.org> och
% <2.debian.pool.ntp.org> eller <1.se.pool.ntp.org> och <2.se.pool.ntp.org>.
% Se <http://www.pool.ntp.org/>.

