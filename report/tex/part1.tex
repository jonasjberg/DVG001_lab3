% ______________________________________________________________________________
%
% DVG001 -- Introduktion till Linux och små nätverk
%                              Inlämningsuppgift #3
% ~~~~~~~~~~~~~~~~~~~~~~~~~~~~~~~~~~~~~~~~~~~~~~~~~
% Author:   Jonas Sjöberg
%           tel12jsg@student.hig.se
%
% Date:     2016-03-15 -- 2016-03-TODO
%
% License:  Creative Commons Attribution 4.0 International (CC BY 4.0)
%           <http://creativecommons.org/licenses/by/4.0/legalcode>
%           See LICENSE.md for additional licensing information.
% ______________________________________________________________________________


\section{Del ett}
Jag gick händelserna i förväg och gjorde under en del moment som inte ingick i
uppgiftsbeskrivningen för den förra laborationen, inlämningsuppgift #2.

Installation och konfiguration av \texttt{sudo} var ett mer eller mindre
nödvändigt steg i förberedelsen av den utvecklingsmiljö som skulle komma att
användas under laborationen.

Den beskrivning som följer är därför delvis hämtad direkt från föregående rapport.


% ______________________________________________________________________________
\subsection{Installation av \texttt{sudo}}
Programmet \texttt{sudo} installeras automatiskt om lösenord för
\texttt{root}-användaren inte specifieras under installationen av
\texttt{Debian}.  Av den anledningen behövde jag själv inte installera
\texttt{sudo}, bara konfiguration var nödvändig.

Från ''Debian GNU\slash Linux Installation Guide''
\cite{debian:install}:

\begin{quotation}
By default you are asked to provide a password for the “root” (administrator)
account and information necessary to create one regular user account. If you do
not specify a password for the “root” user this account will be disabled but
the sudo package will be installed later to enable administrative tasks to be
carried out on the new system. 
\end{quotation}


Programmet \texttt{sudo} ligger i paketet med samma namn.
För att ta redo på vilket paket som innehåller en viss fil kan man använda
kommandot i Programlistning~\ref{listing:dpkg-search}.

\begin{listing}[H]
\shellcode{tex/dpkg-search.sh}
\caption{Kommando för att söka efter filer i paket.}
\label{listing:dpkg-search}
\end{listing}


% ______________________________________________________________________________
\subsection{Konfiguration av \texttt{sudo}}
För att kunna installera program och sköta andra administrativa sysslor
konfigureras \texttt{sudo} enligt följande:

\begin{enumerate}
  \item Användaren läggs till i \texttt{sudo}-gruppen med kommandot:

        \texttt{\$ adduser jonas sudo}

  \item Konfigurationsfilen för \texttt{sudo} ändras. Genom att köra
        följande kommandon så öppnar en särskild texteditor som tillåter en
        säkrare mijö för att ändra i filen \texttt{/etc/sudoers}:
        
        \texttt{\$ su} \\
        \texttt{\$ visudo}

        Med texteditorn lägger man till raden:
        \texttt{jonas ALL=(ALL:ALL) ALL}

        Det ger användaren \texttt{jonas} möjlighet att köra alla kommandon som 
        administratör med hjälp av \texttt{sudo}.
\end{enumerate}


% ______________________________________________________________________________
\subsection{Installation av \texttt{ntp}}
Här installeras programvara för att synkronisera olika datorers
tidsinställningar.  Det görs med \texttt{Debian}-paketet \texttt{ntp} som
ställer datorns klocka efter atomur som görs tillgängliga över internet genom
särskilda servrar.


% ______________________________________________________________________________
\subsubsection{Sökning i paketlistor med \texttt{apt}}
För att söka efter paketet \texttt{ntp} i lokala listor över paketförrådens
innehåll används kommandot i Programlistning~\ref{listing:ntp-apt-search},
som dessutom använder \texttt{grep} för att bara visa rader som börjar med
''ntp''.

\begin{listing}[H]
\shellcode{tex/ntp-apt-search.sh}
\caption{Kommando för att söka i lokala paketlistor efter textsträngar.}
\label{listing:ntp-apt-search}
\end{listing}


% ______________________________________________________________________________
\subsubsection{Installation av programmet \texttt{ntpdate} och \texttt{ntp}}
Programmen installeras sedan genom att köra kommandot i
Programlistning~\ref{listing:ntp-install}, som också visar information som skrivs
ut av \texttt{apt} under installationsprocessen.

\begin{listing}[H]
\shellcode{tex/ntp-install.sh}
\caption{Kommando för att installera programmen \texttt{ntp} och \texttt{ntpdate}.}
\label{listing:ntp-install}
\end{listing}


% ______________________________________________________________________________
\subsubsection{Sökning i paketlistor med \texttt{aptitude}}
Användning av \texttt{apt}, \texttt{aptitude} eller
\texttt{dpkg} är många gånger utbytbar, både \texttt{apt} och \texttt{aptitude}
är abstraktioner byggda ovanpå \texttt{dpkg}. Även om \texttt{aptitude} ska vara
det nyaste och mest användarvänliga, föredrar användare många gånger ändå
att använda \texttt{apt}, av många olika skäl \cite{superuser:aptitude-apt}.

Efter installationen kan ett sökkommando motsvarande det i
Programlistning~\ref{listing:ntp-apt-search} köras, fast den här gången med
\texttt{aptitude}, som erbjuder inbyggd filtrering av resultat. Här visas
träffar som börjar med ''ntp''. Resultatet visas i
Programlistning~\ref{listing:ntp-aptitude-search}.

\begin{listing}[H]
\shellcode{tex/ntp-aptitude-search.sh}
\caption{Kommando för att söka bland installerade paket med \texttt{aptitude}.}
\label{listing:ntp-aptitude-search}
\end{listing}


% ______________________________________________________________________________
\subsubsection{Konfiguration av \texttt{ntp}}
Programmen \texttt{ntpdate} och \texttt{ntp} konfigureras så att de använder
samma atomur-server på internet från ''ntp-pool''-projektet, exempelvis
\url{1.debian.pool.ntp.org} och \url{2.debian.pool.ntp.org} eller
\url{1.se.pool.ntp.org} och \url{2.se.pool.ntp.org}.
% Se <http://www.pool.ntp.org/>.

Efter installationen av programmen kan information som vanligt läsas från
manualsidorna. Konfigurationsfilen \texttt{/etc/ntp.conf} innehåller
standardinställningar, däribland en lista med servrar som används vid
tidssynkronisering. Ett utdrag visas i Programlistning~\ref{listing:ntp-conf}.

\begin{listing}[H]
\configfile[firstline=18,lastline=24]{tex/ntp-config-default}
\caption{Utdrag ur den omodifierade konfigurationsfilen för \texttt{ntp}.}
\label{listing:ntp-conf}
\end{listing}

Nämnvärt är ordet \texttt{iburst} som är en \texttt{command option} som 
beskrivs i manualsidan för \texttt{ntp.conf(5)}:

\begin{quotation}
When  the  server  is  unreachable, send a burst of eight packets
instead of the usual one.  The packet spacing is normally 2 s; however, the
spacing between the first and second packets can be changed with the calldelay
command to allow additional time for a modem or ISDN call to  complete.   This
option is valid with only the server command and is a recommended option with
this command.
\end{quotation}


Raderna kan ändras för att innehålla de servrar vi själva vill använda,
förslagsvis svenska servrar vilket kan antas ge bättre tillförlitlighet och
snabbare respons. 
På ''ntp pool''-projektets sida \url{http://www.pool.ntp.org/zone/se} finns en
lista över svenska servrar tillsammans med statistik över deras aktivitet.
Enligt instruktioner \cite{ntp:pool-use} läggs rader till i
konfigurationsfilen, som då får utseendet som visas i
Programlistning~\ref{listing:ntp-conf-mod}.  Ändringarna görs med texteditorn
\texttt{vim} som körs med högre rättigheter med hjälp av \texttt{sudo}.

\begin{listing}[H]
\configfile[firstline=18,lastline=28]{tex/ntp-config-modified}
\caption{Utdrag ur konfigurationsfilen för \texttt{ntp} efter inkludering av
servrar från en svensk ''pool zone''.}
\label{listing:ntp-conf-mod}
\end{listing}


Efter att konfigurationsfilen har modifierats startas \texttt{ntpd} om
\cite{ubuntu:ntp-serverguide}, varpå en lista med ''peers'' som servern känner
till skrivs ut. Detta visas i Programlistning~\ref{listing:ntp-reload-list}.

\begin{listing}[H]
\shellcode{tex/ntp-reload-list}
\caption{Omstart av \texttt{ntpd} och listning av ''peers''.}
\label{listing:ntp-reload-list}
\end{listing}


% ______________________________________________________________________________
\subsubsection{Konfiguration av \texttt{ntpdate}}
Konfigurationsfilen för \texttt{ntpdate} finns i sökvägen \texttt{/etc/default/ntpdate}
och ett utdrag ur dess innehåll visas i Programlistning~\ref{listing:ntpdate-conf}.

\begin{listing}[H]
\configfile[]{tex/ntpdate-config-default}
\caption{Konfigurationsfilen för \texttt{ntpdate}.}
\label{listing:ntpdate-conf}
\end{listing}

På rad~6 ser vi att \texttt{ntpdate} är inställt att hämta listan över
servrar från konfigurationsfilen för \texttt{ntp}, som redan har modifierats.
