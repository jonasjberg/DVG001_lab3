% ______________________________________________________________________________
%
% DVG001 -- Introduktion till Linux och små nätverk
%                              Inlämningsuppgift #3
% ~~~~~~~~~~~~~~~~~~~~~~~~~~~~~~~~~~~~~~~~~~~~~~~~~
% Author:   Jonas Sjöberg
%           tel12jsg@student.hig.se
%
% Date:     2016-03-15 -- 2016-03-20
%
% License:  Creative Commons Attribution 4.0 International (CC BY 4.0)
%           <http://creativecommons.org/licenses/by/4.0/legalcode>
%           See LICENSE.md for additional licensing information.
% ______________________________________________________________________________


\section{Del två}


% ______________________________________________________________________________
\subsection{Skapa en användare}
Här skapas en ny användare med kommandot \texttt{adduser}.  Skapandet av
användaren ''gibson'' visas i Programlistning~\ref{listing:adduser-gibson}.

\begin{listing}[H]
\shellcode{tex/adduser-gibson}
\caption{Skapande av en ny användare ''gibson''.}
\label{listing:adduser-gibson}
\end{listing}

Min katt Gibson har inte ett eget rum och använder för tillfället inte heller
någon telefon. En del av fälten lämnas därför tomma.


% ______________________________________________________________________________
\subsection{Logga in som den nya användaren}
Efter att den nya användaren har skapats så loggar man in som den nya
användaren.  

\subsubsection{Testkörning av \texttt{sudo} med den nya användaren}
En testkörning av \texttt{ifconfig} med \texttt{sudo} visar också
att användaren ''gibson'' inte har möjlighet att köra program med högre
privilegier ännu. Körningen visas i Programlistning~\ref{listing:su-gibson}.

\begin{listing}[H]
\shellcode{tex/su-gibson}
\caption[Logga in med annan användare]{Inloggning och testkörning av
         \texttt{sudo} med användaren ''gibson''.}
\label{listing:su-gibson}
\end{listing}


% ______________________________________________________________________________
\subsubsection{Lägga till den nya användaren i gruppen \texttt{sudo}}
För att ge den nya användaren möjlighet att köra \texttt{sudo} läggs den till i
gruppen som kallas \texttt{sudoers}. Detta visas i
Programlistning~\ref{listing:sudoers-gibson} tillsammans med en testkörning för
att verifiera att användaren fått möjlighet att köra kommandon som
\texttt{root} genom \texttt{sudo}.

\begin{listing}[H]
\shellcode{tex/sudoers-gibson}
\caption[Lägga till användare i gruppen \texttt{sudoers}]
        {Här inkluderas användaren ''gibson'' i gruppen \texttt{sudoers},
         följt av en testkörning.}
\label{listing:sudoers-gibson}
\end{listing}


Användaren ''gibson'' kan nu köra programmet \texttt{ifconfig} utan problem.
