% ______________________________________________________________________________
%
% DVG001 -- Introduktion till Linux och små nätverk
%                              Inlämningsuppgift #3
% ~~~~~~~~~~~~~~~~~~~~~~~~~~~~~~~~~~~~~~~~~~~~~~~~~
% Author:   Jonas Sjöberg
%           tel12jsg@student.hig.se
%
% Date:     2016-03-15 -- 2016-03-TODO
%
% License:  Creative Commons Attribution 4.0 International (CC BY 4.0)
%           <http://creativecommons.org/licenses/by/4.0/legalcode>
%           See LICENSE.md for additional licensing information.
% ______________________________________________________________________________


\section{Genomförande}
% Här skriver ni vilka steg ni gjorde och resultatet av dem. Ni skall ha med
% information så att vi kan se hur ni har gjort, dvs beskrivande text,
% skärmdumpar, bilder etc.  Längre skärmdumpar, innehåll i relevanta filer och
% större bilder lägger ni i bilagor, som bilaga I, så att de inte tar över en
% sida själva.
% Kommandon som ni skriver i ett skal skall skrivas i detta format, som är
% teckenformatmall ”Exempel” i OpenOffice/LibreOffice. Detta så att de
% skiljer sig från övriga brödtext i stycket.  Detta underlättar läsningen för
% andra, som oss lärare.




% ______________________________________________________________________________
\subsection{Inledande åtgärder}

\subsubsection{Montera den delade mappen automatiskt}
Den delade mappen som konfigurerats tidigare under installationen av
gästsystemet har hittils monterats manuellt innan varje användning med
kommandot i Programlistning~\ref{listing:mount-share-manual}.

\begin{listing}[H]
  \shellcode{tex/mount-share-manual.sh}
  \caption{Kommando för att montera den delade mappen}
  \label{listing:mount-share-manual}
\end{listing}

Istället för att behöva göra det manuellt varje gång kan mappen läggas till i
\texttt{/etc/fstab} och monteras automatiskt vid systemstart
\cite{virtualbox:sharemount}.

Först görs en säkerhetskopia av \texttt{/etc/fstab} enligt
Programlistning~\ref{listing:mount-fstab-backup}.



% Byt nätverkstyp i VirtualBox till bridged
% Kolla ip-adress, ifconfig deprecated (?)
% $ ip a 
% ...


