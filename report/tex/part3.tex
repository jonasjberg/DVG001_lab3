% ______________________________________________________________________________
%
% DVG001 -- Introduktion till Linux och små nätverk
%                              Inlämningsuppgift #3
% ~~~~~~~~~~~~~~~~~~~~~~~~~~~~~~~~~~~~~~~~~~~~~~~~~
% Author:   Jonas Sjöberg
%           tel12jsg@student.hig.se
%
% Date:     2016-03-15 -- 2016-03-TODO
%
% License:  Creative Commons Attribution 4.0 International (CC BY 4.0)
%           <http://creativecommons.org/licenses/by/4.0/legalcode>
%           See LICENSE.md for additional licensing information.
% ______________________________________________________________________________


\section{Del tre}


% ______________________________________________________________________________
\subsection{Skapa katalogstruktur}
% TODO: Skapa användare, se till att användaren kan använda sudo.


% ______________________________________________________________________________
\subsection{Exekvering av kommandon}
% TODO: ..
% Skriv de kommandon som använts för att skapa filer, kataloger, rättigheter. 
%
% Förklara vad varje kommandorad gör och varför det händer, 
% samt vilken användare som utför delkommandon.
%
% Förklara felmeddelanden. 
%
% Katalogstrukturen redovisas lämpligen med kommandot `tree(1)` med växeln `-a`.


\subsubsection{Installation av programmet \texttt{ntpdate} och \texttt{ntp}}
% TODO: ..

\subsubsection{Konfiguration av \texttt{ntpdate} och \texttt{ntp}}
% TODO: ..
%
% Konfigurera dem så att de använder samma atomur-server på internet från
% ntp-pool-projektet, exempelvis <1.debian.pool.ntp.org> och
% <2.debian.pool.ntp.org> eller <1.se.pool.ntp.org> och <2.se.pool.ntp.org>.
% Se <http://www.pool.ntp.org/>.



