% ______________________________________________________________________________
%
% DVG001 -- Introduktion till Linux och små nätverk
%                              Inlämningsuppgift #3
% ~~~~~~~~~~~~~~~~~~~~~~~~~~~~~~~~~~~~~~~~~~~~~~~~~
% Author:   Jonas Sjöberg
%           tel12jsg@student.hig.se
%
% Date:     2016-03-15 -- 2016-03-TODO
%
% License:  Creative Commons Attribution 4.0 International (CC BY 4.0)
%           <http://creativecommons.org/licenses/by/4.0/legalcode>
%           See LICENSE.md for additional licensing information.
% ______________________________________________________________________________


\section{Del tre}


% ______________________________________________________________________________
\subsection{Skapa katalogstruktur}
Här ska rättigheter för användare demonstreras genom skapande av en uppsättning filer
och kataloger. Den katalogstruktur som efterfrågas visas i Tabell~\ref{table:tree}.

\begin{table}[]
  \centering
  \caption{Efterfrågad katalogstruktur}
  \label{table:tree}
  \begin{tabular}{@{}llll@{}}
    \toprule
            Sökväg         & Rättigheter                   & Ägare                     & Grupp               \\ \midrule
    \texttt{/tmp/del3}     & \texttt{drwxr-xr-x}           & Användaren från del två   & Gruppen som användaren tillhör \\
    \texttt{/tmp/del3/a1}  & \texttt{drwx-{}-{}-{}-{}-{}-} & \texttt{root}             & Gruppen som användaren tillhör \\
    \texttt{/tmp/del3/a2}  & \texttt{-rwxr-{}-r-{}-}       &  Användaren från del två  & \texttt{root}                  \\
    \texttt{/tmp/del3/a3}  & \texttt{drwxr-{}-r-{}-}       &  Användaren från del två  & Gruppen som användaren tillhör \\
    \texttt{/tmp/del3/a4}  & \texttt{-rwxrwx-{}-{}-}       & \texttt{root}             & \texttt{root}                  \\\bottomrule
  \end{tabular}
\end{table}


% ______________________________________________________________________________
\subsubsection{Skapa \texttt{/tmp/del3/a1}}
Katalogen skapas enligt Programlistning~\ref{listing:part3-del3a1}.

\begin{listing}[H]
\shellcode{tex/part3-del3a1}
\caption{Skapandet av \texttt{/tmp/del3/a1}}
\label{listing:part3-del3a1}
\end{listing}

% ______________________________________________________________________________
\subsubsection{Skapa \texttt{/tmp/del3/a2}}
Filen skapas enligt Programlistning~\ref{listing:part3-del3a2}.

\begin{listing}[H]
\shellcode{tex/part3-del3a2}
\caption{Skapandet av \texttt{/tmp/del3/a2}}
\label{listing:part3-del3a2}
\end{listing}

% ______________________________________________________________________________
\subsubsection{Skapa \texttt{/tmp/del3/a3}}
Katalogen skapas enligt Programlistning~\ref{listing:part3-del3a3}.

\begin{listing}[H]
\shellcode{tex/part3-del3a3}
\caption{Skapandet av \texttt{/tmp/del3/a3}}
\label{listing:part3-del3a3}
\end{listing}

% ______________________________________________________________________________
\subsubsection{Skapa \texttt{/tmp/del3/a4}}
Filen skapas enligt Programlistning~\ref{listing:part3-del3a4}.

\begin{listing}[H]
\shellcode{tex/part3-del3a4}
\caption{Skapandet av \texttt{/tmp/del3/a4}}
\label{listing:part3-del3a4}
\end{listing}

% ______________________________________________________________________________
\subsubsection{Verifiering av skapade filer i \texttt{/tmp/del3}}
För att kontrollera att de skapade filerna och katalogerna har de egenskaper
som efterfrågats körs programmet i Programlistning~\ref{listing:part3-verify}.
Programmet loopar över innehållet i katalogen \texttt{/tmp/del3} och katalogen
själv med en möjligen onödigt komplex parameter-expansion \texttt{{,*} som
expanderas till .. % TODO: .....

\begin{listing}[H]
\shellcode{tex/part3-verify}
\caption{Verifiering av skapade filer i \texttt{/tmp/del3}}
\label{listing:part3-verify}
\end{listing}




% ______________________________________________________________________________
\subsection{Exekvering av kommandon}
% TODO: ..
% Skriv de kommandon som använts för att skapa filer, kataloger, rättigheter. 
%
% Förklara vad varje kommandorad gör och varför det händer, 
% samt vilken användare som utför delkommandon.
%
% Förklara felmeddelanden. 
%
% Katalogstrukturen redovisas lämpligen med kommandot `tree(1)` med växeln `-a`.


