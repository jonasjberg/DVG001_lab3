% ______________________________________________________________________________
%
% DVG001 -- Introduktion till Linux och små nätverk
%                              Inlämningsuppgift #3
% ~~~~~~~~~~~~~~~~~~~~~~~~~~~~~~~~~~~~~~~~~~~~~~~~~
% Author:   Jonas Sjöberg
%           tel12jsg@student.hig.se
%
% Date:     2016-03-15 -- 2016-03-20
%
% License:  Creative Commons Attribution 4.0 International (CC BY 4.0)
%           <http://creativecommons.org/licenses/by/4.0/legalcode>
%           See LICENSE.md for additional licensing information.
% ______________________________________________________________________________


\section{Del tre}


% ______________________________________________________________________________
\subsection{Skapa katalogstruktur}
Här ska rättigheter för användare demonstreras genom skapande av en uppsättning filer
och kataloger. Den katalogstruktur som efterfrågas visas i Tabell~\ref{table:tree}.

\begin{table}[]
  \centering
  \caption{Efterfrågad katalogstruktur}
  \label{table:tree}
  \begin{tabular}{@{}llll@{}}
  \toprule
          Sökväg        & Rättigheter                   & Ägare                    & Grupp                          \\ \midrule
  \texttt{/tmp/del3}    & \texttt{drwxr-xr-x}           & Användaren från del två  & Gruppen som användaren tillhör \\
  \texttt{/tmp/del3/a1} & \texttt{drwx-{}-{}-{}-{}-{}-} & \texttt{root}            & Gruppen som användaren tillhör \\
  \texttt{/tmp/del3/a2} & \texttt{-rwxr-{}-r-{}-}       &  Användaren från del två & \texttt{root}                  \\
  \texttt{/tmp/del3/a3} & \texttt{drwxr-{}-r-{}-}       &  Användaren från del två & Gruppen som användaren tillhör \\
  \texttt{/tmp/del3/a4} & \texttt{-rwxrwx-{}-{}-}       & \texttt{root}            & \texttt{root}                  \\\bottomrule
  \end{tabular}
\end{table}



Katalogen \texttt{/tmp/del3/a1} skapas enligt Programlistning~\ref{listing:part3-del3a1}.

\begin{listing}[H]
\shellcode{tex/part3-del3a1}
\caption{Skapandet av \texttt{/tmp/del3/a1}}
\label{listing:part3-del3a1}
\end{listing}


Filen \texttt{/tmp/del3/a2} skapas enligt Programlistning~\ref{listing:part3-del3a2}.

\begin{listing}[H]
\shellcode{tex/part3-del3a2}
\caption{Skapandet av \texttt{/tmp/del3/a2}}
\label{listing:part3-del3a2}
\end{listing}


Katalogen \texttt{/tmp/del3/a3} skapas enligt Programlistning~\ref{listing:part3-del3a3}.

\begin{listing}[H]
\shellcode{tex/part3-del3a3}
\caption{Skapandet av \texttt{/tmp/del3/a3}}
\label{listing:part3-del3a3}
\end{listing}


Filen \texttt{/tmp/del3/a4} skapas enligt Programlistning~\ref{listing:part3-del3a4}.

\begin{listing}[H]
\shellcode{tex/part3-del3a4}
\caption{Skapandet av \texttt{/tmp/del3/a4}}
\label{listing:part3-del3a4}
\end{listing}


% ______________________________________________________________________________
\subsubsection{Verifiering av de filer som skapats i \texttt{/tmp/del3}}
För att kontrollera att de skapade filerna och katalogerna har de egenskaper
som efterfrågats körs programmet i Programlistning~\ref{listing:part3-verify}.
Programmet loopar över både innehållet i katalogen \texttt{/tmp/del3} och
katalogen själv med en möjligen onödigt komplex parameter-expansion
\texttt{\{,*\}} som expanderas till katalogen och dess underkataloger.  Dessa
skickas som argument till \texttt{ls} vars utskrift slutligen kolumniseras med
\texttt{column}.

\begin{listing}[H]
\shellcode{tex/part3-verify.sh}
\caption{Verifiering av de kataloger och filer som skapats}
\label{listing:part3-verify}
\end{listing}

Resultatet i Programlistning~\ref{listing:part3-verify} matchar den
struktur som efterfrågas i Tabell~\ref{table:tree}.


% ______________________________________________________________________________
\subsubsection{Skript för skapande av katalogstruktur}
Eftersom att katalogstrukturen ligger i \texttt{/tmp/} så är det stor risk att
den raderas när systemet startar om. Ovanstående kommandon samlas i ett skript
så att katalogstrukturen enkelt kan återskapas efter en omstart.

Skriptet visas i Programlistning~\ref{listing:part3-setup} och körning visas i
Programlistning~\ref{listing:part3-setup-output}.

\begin{listing}[H]
\shellcode{tex/part3-setup.sh}
\caption[Skript för att skapa katalogstruktur]{Skript som körs för att skapa
         filer och kataloger med särskilda rättigheter.}
\label{listing:part3-setup}
\end{listing}

\begin{listing}[H]
\shellcode{tex/part3-setup-output}
\caption{Körning av skriptet i Programlistning~\ref{listing:part3-setup}.}
\label{listing:part3-setup-output}
\end{listing}


% ______________________________________________________________________________
\subsection{Exekvering av kommandon}
% Förklara vad varje kommandorad gör och varför det händer, 
% samt vilken användare som utför delkommandon.
%
% Förklara felmeddelanden. 
%
% Katalogstrukturen redovisas lämpligen med kommandot `tree(1)` med växeln `-a`.

\newcommand{\explainedcmd}[4]{
\begin{listing}[H]
\shellcode[firstline={#1},lastline={#2}]{tex/part3-commands}
\caption[#3]{#4}
\label{listing:ntp-conf-mod}
\end{listing}
}

Under de Programlistningar som följer står förklaringar av vad varje kommando
gör och varför det händer skrivna under respektive Programlistning.

\explainedcmd{1}{3}
             {Kommandot \texttt{touch /tmp/del3/a1/f1}}
             {Kommandot misslyckas på grund av att enbart ägaren \texttt{root}
              har behörighet att skriva i målkatalogen.  
              Vid den andra körningen används \texttt{sudo} för att anta 
              \texttt{roots} rättigheter och kommandot lyckas.}

\explainedcmd{4}{6}
             {Kommandot \texttt{sudo echo "Hello World" | tee /tmp/del3/a3/f1}}
             {Körningen av \texttt{tee} misslyckas eftersom att ''gibson''
              saknar rättigheter. Bara \texttt{echo} körs med \texttt{sudo},
              rättigheterna ''nollställs'' då datan passerar pipe-symbolen,
              \texttt{|}.}

\explainedcmd{7}{8}
             {Kommandot \texttt{sudo echo "Hello World" | tee /tmp/del3/a2}}
             {Körningen av \texttt{tee} lyckas och ''Hello World'' skrivs
              till filen \texttt{a2}, som ägs av ''gibson''.}

\explainedcmd{9}{11}
             {Kommandot \texttt{sudo echo "Hello World" | tee /tmp/del3/a4}}
             {Körningen av \texttt{tee} misslyckas eftersom att ''gibson''
              saknar rättigheter för att skriva till filen \texttt{a4}. 
              Enbart \texttt{echo} körs med \texttt{sudo}, rättigheterna
              ''nollställs'' då datan passerar pipe-symbolen, \texttt{|}.}

\explainedcmd{12}{13}
             {Kommandot \texttt{cat /tmp/del3/a2}}
             {Innehållet i filen \texttt{a2} skrivs ut.}

\explainedcmd{14}{15}
             {Kommandot \texttt{sudo cat /tmp/del3/a2}}
             {Innehållet i filen \texttt{a2} skrivs ut. Användningen av
              \texttt{sudo} är överflödig, ''gibson'' har redan rättighet att
              läsa filen.}

\explainedcmd{16}{17}
             {Kommandot \texttt{cat /tmp/del3/a4}}
             {Körningen misslyckas eftersom att filen \texttt{a4} ägs av
              \texttt{root} och att ''gibson'' är inte medlem i gruppen 
              \texttt{root} och saknar därmed rättigheter.}

\explainedcmd{18}{18}
             {Kommandot \texttt{sudo cat /tmp/del3/a4}}
             {Körningen lyckas men filen \texttt{a4} är tom då kommandot för
              att skriva till filen misslyckades. Den text som skrevs ut för
              att beskriva felet skrevs till \texttt{stderror}, medan 
              \texttt{stdout} som skrevs till filen, var tomt.}

\explainedcmd{19}{20}
             {Kommandot \texttt{echo "Goodbye World" | sudo tee /tmp/del3/a2}}
             {Körningen lyckas trots att ''gibson'' inte har rättighet att
              skriva till \texttt{a2} då kommandot körs som \texttt{root} med
              \texttt{sudo}.}

\explainedcmd{21}{22}
             {Kommandot \texttt{cat /tmp/del3/a2}}
             {Körningen lyckas, ''gibson'' har rättighet att läsa \texttt{a2}.}

\explainedcmd{23}{24}
             {Kommandot \texttt{sudo cat /tmp/del3/a2}}
             {Körningen lyckas, ''gibson'' har rättighet att läsa \texttt{a2}.
              Användningen av \texttt{sudo} är överflödig.}

\explainedcmd{25}{26}
             {Kommandot \texttt{sudo rm -r /tmp/del3/}}
             {Hela katalogstrukturen tas bort.}

\subsection{Redovisning av resultat}
För att kontrollera och redovisa katalogstrukturens innehåll används ännu ett
skript som visas i Programlistning~\ref{listing:part3-treeish}.

Exekvering visas i Programlistning~\ref{listing:part3-treeish-output}. Kataloger
och filer skrivs ut och märks i den vänstra kolumnen, filers innehåll skrivs ut
i den högra kolumnen. Notera att det sista kommandot \texttt{sudo rm -r /tmp/del3/}
har exkluderats för att det ska vara möjligt att titta på resultatet.

\begin{listing}[H]
\shellcode{tex/treeish.sh}
\caption[Skript för att skapa katalogstruktur med rättigheter]{Skript som körs
         för att skapa filer och kataloger med särskilda rättigheter.}
\label{listing:part3-treeish}
\end{listing}

\begin{listing}[H]
\shellcode{tex/treeish-output}
\caption{Körning av skriptet i Programlistning~\ref{listing:part3-treeish}.}
\label{listing:part3-treeish-output}
\end{listing}
