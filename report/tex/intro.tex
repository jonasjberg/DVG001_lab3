% ______________________________________________________________________________
%
% DVG001 -- Introduktion till Linux och små nätverk
%                              Inlämningsuppgift #3
% ~~~~~~~~~~~~~~~~~~~~~~~~~~~~~~~~~~~~~~~~~~~~~~~~~
% Author:   Jonas Sjöberg
%           tel12jsg@student.hig.se
%
% Date:     2016-03-15 -- 2016-03-TODO
%
% License:  Creative Commons Attribution 4.0 International (CC BY 4.0)
%           <http://creativecommons.org/licenses/by/4.0/legalcode>
%           See LICENSE.md for additional licensing information.
% ______________________________________________________________________________


\section{Inledning}\label{inledning}
% Skriv en kort inledning här som beskriver kortfattat vad rapporten handlar
% om. Den skall vara orienterande om Bakgrund och Syfte.


% ______________________________________________________________________________
\subsection{Bakgrund}
% Beskriv lite mer ingående om bakgrunden till uppgiften, vad den handlar om.


% ______________________________________________________________________________
\subsection{Syfte}
% Skriv lite mer ingående om syftet med uppgiften.

% ______________________________________________________________________________
%\subsection{Nomenklatur}


% ~~~~~~~~~~~~~~~~~~~~~~~~~~~~~~~~~~~~~~~~~~~~~~~~~~~~~~~~~~~~~~~~~~~~~~~~~~~~~~
\section{Planering}
% Kort och orienterande om hur ni tänkte genomföra uppgiften.
% Orienterande om Planering och Genomförande.


% ______________________________________________________________________________
\subsection{Arbetsmetod}
% Hur kommer ni att arbeta?  Detta är en lite längre text än den rent
% orienterande texten i Planering och genomförande ovan.

Nedan följer en preliminär redogörelse för den experimentuppställning som används
under laborationen:

\begin{itemize}
  \item Laborationen utförs på en \texttt{ProBook-6545b} laptop som kör
        \texttt{Xubuntu 15.10} på kerneln \texttt{Linux 3.19.0-28-generic}.

  \item Rapporten skrivs i \LaTeX\  som kompileras till pdf med \texttt{latexmk}.

  \item Både rapporten och koden skrivs med texteditorn \texttt{Vim}.

  \item För versionshantering av både rapporten och programkod används \texttt{Git}.
    \begin{itemize}
      \item Källkod till programmet och rapporten finns att hämta på:

            \url{https://github.com/jonasjberg/DVG001\_lab3}

      \item Hämta hem repon genom att exekvera följande från kommandoraden:
            
            \texttt{git clone git@github.com:jonasjberg/DVG001\_lab3.git}

    \end{itemize}

  \item Virtualisering sker med \texttt{Oracle VirtualBox} version
        \texttt{5.0.10\_Ubuntu r104061}.
\end{itemize}

